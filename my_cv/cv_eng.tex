\documentclass[a4paper, 10pt]{article}
%\documentclass{article}
\usepackage [utf8] {inputenc}
% включаем переносы для русского и английского языка
\usepackage[english,russian]{babel}
% Начинать первый параграф раздела следует с красной строки
\usepackage{indentfirst}
% Выбор внутренней TEX−кодировки
\usepackage [T2A]{fontenc}

\usepackage{cmap}
\usepackage{multirow}

\usepackage{xcolor}

\usepackage{hyperref}
\definecolor{LINKCOLOUR}{rgb}{0.1,0.0,0.9}
\hypersetup{colorlinks,breaklinks,urlcolor=LINKCOLOUR,linkcolor=LINKCOLOUR}

\usepackage{geometry} % Меняем поля страницы
\geometry{left=20mm}% левое поле
\geometry{right=15mm}% правое поле
\geometry{top=15mm}% верхнее поле
\geometry{bottom=15mm}% нижнее поле

\usepackage{titlesec} % Used to customize the \section command
\titleformat{\section}{\Large\raggedright}{\color{green}}{0em}{}[\titlerule] % Text formatting of sections
%\titleformat{\section}{\large\scshape\raggedright}{\color{green}}{0em}{}[\titlerule] % Text formatting of sections
\titlespacing{\section}{2pt}{3pt}{3pt} % Spacing around sections

\usepackage{longtable}
\usepackage{enumitem}

\pagestyle{empty}

\begin{document}

{\LARGE\textbf{Alexander Kindyakov}}

\begin{flushright}
    {\itshape
        Moscow, Russia \\
        +7--926--453--25--30 \\
        \href{mailto:akindyakov@gmail.com}{akindyakov@gmail.com} \\
        \href{https://github.com/akindyakov}{github.com/akindyakov} \\
        \href{http://ru.linkedin.com/in/alexanderkindyakov}{ru.linkedin.com/in/alexanderkindyakov} \\
    }
\end{flushright}


\section{Experience}
\begin{longtable}{p{20mm}|p{140mm}}
2013 ---
& \textbf{``Yandex''} \\
& \href{https://yandex.ru/}{yandex.ru} \\
& Software Engineer \\
& \begin{itemize}[topsep = 0pt, itemsep = 0pt]
    \item[+] Web--Search antispam.
    \item[+] Distributed key--value in-memory fast data base (cluster size 50-100 nodes):
        \begin{itemize}[topsep = 0pt, itemsep = 0pt]
            \item logging in syslog and logfile instead of stderr;
            \item revise response formats code, create JSON format and sending detailed error report;
            \item stability and exceptions--safety problem (there is no chance to fail in case of a bad request or other manageable problem);
            \item solve replicas synchronization problem in the case of pool connectivity (bug in renovated replica update process by HEAD replica);
            \item generating and computing data sets for different instances of this DB;
            \item monitorings for instance updating processes.
        \end{itemize}

    \item[+] High-performance web-documents processing system:
        \begin{itemize}[topsep = 0pt, itemsep = 0pt]
            \item refactoring base library (for the sake of optimization)
                \begin{itemize}[topsep = 0pt, itemsep = 0pt]
                    \item carry over serializing/deserializing in the separate library, hide implementations behind universal interface;
                    \item use hash-table for intermediate results instead of slow custom solution based on the std::vector;
                \end{itemize}
            \item creating code unit and integration tests;
            \item monitorings for all point of the workflow;
            \item carry over all computations from cluster (size 70-90 nodes) solution to Map--Reduce system (Yandex implementation of Map-Reduce framework for big data);
            \item responsibility for the whole process from C++ code and text features code to computed values delivery to web-search back-end.
        \end{itemize}
    \item[+] Declarative language to describe html-text features:
    \begin{itemize}[topsep = 0pt, itemsep = 0pt]
        \item ajust new text analyzers;
        \item revise code solution for serializing compiled byte-code (using google protoBuf instead of plain text);
    \end{itemize}


\end{itemize}
\\

2011 --- 2013 & \textbf{``3Detection Labs''} \\
& \href{http://www.3detection.com/}{www.3detection.com} \\
& Software Engineer \\
& \begin{itemize}[topsep = 0pt, itemsep = 0pt]
    \item[+] Automatic control system for robotic 3--degree--of--freedom manipulator.
    \item[+] Testing, adjustment and problem diagnostics software for robotic arm.
    \item[+] Robot navigation system based on the telemetry, GPS and Earth magnetic field.
    \item[+] Remote control and adjustment system for mobile robot.
    \item[+] Image recognition for mobile robot computer vision.
    \item[+] AI for mobile robot (project was scrapped).
\end{itemize}
\\
\end{longtable}

\section{Education}
\begin{longtable}{p{20mm}|p{140mm}}
2008 --- 2014
& ``BMSTU'' \\
& \href{http://www.bmstu.ru}{www.bmstu.ru} \\
& Mechatronics, Robotics, and Automation Engineering \\
& Master's degree \\
& Diploma with honours \\
& Graduation project: ``\textit{Space satellite manipulator precise drive}'' \\
& Research advisor: Boshlyakov Andrey \\
\end{longtable}

\section{Language proficiency}
\begin{longtable}{p{20mm}|p{140mm}}
& \begin{itemize}[topsep = 0pt, itemsep = 0pt]
    \item[+] Russian --- native
    \item[+] English --- intermediate
\end{itemize} \\
\end{longtable}

\end{document}

