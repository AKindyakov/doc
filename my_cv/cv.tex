\documentclass[a4paper, 10pt]{article}
%\documentclass{article}
\usepackage [utf8] {inputenc}
% включаем переносы для русского и английского языка
\usepackage[english,russian]{babel}
% Начинать первый параграф раздела следует с красной строки
\usepackage{indentfirst}
% Выбор внутренней TEX−кодировки
\usepackage [T2A]{fontenc}

\usepackage{cmap}
\usepackage{multirow}

\usepackage{xcolor}

\usepackage{hyperref}
\definecolor{LINKCOLOUR}{rgb}{0.1,0.0,0.9}
\hypersetup{colorlinks,breaklinks,urlcolor=LINKCOLOUR,linkcolor=LINKCOLOUR}

\usepackage{geometry} % Меняем поля страницы
\geometry{left=20mm}% левое поле
\geometry{right=15mm}% правое поле
\geometry{top=15mm}% верхнее поле
\geometry{bottom=15mm}% нижнее поле

\usepackage{titlesec} % Used to customize the \section command
\titleformat{\section}{\Large\raggedright}{\color{green}}{0em}{}[\titlerule] % Text formatting of sections
%\titleformat{\section}{\large\scshape\raggedright}{\color{green}}{0em}{}[\titlerule] % Text formatting of sections
\titlespacing{\section}{2pt}{3pt}{3pt} % Spacing around sections

\usepackage{longtable}
\usepackage{enumitem}

\pagestyle{empty}

\begin{document}

{\LARGE\textbf{Александр Киндяков}}

\begin{flushright}
    {\itshape
        г.Москва, Россия \\
        +7--926--453--25--30 \\
        \href{mailto:akindyakov@gmail.com}{akindyakov@gmail.com} \\
        \href{https://github.com/akindyakov}{github.com/akindyakov} \\
        \href{http://ru.linkedin.com/in/alexanderkindyakov}{ru.linkedin.com/in/alexanderkindyakov} \\
    }
\end{flushright}


\section{Опыт работы }
\begin{longtable}{p{20mm}|p{140mm}}
2013 ---
& \textbf{``Яндекс''} \\
& \href{https://yandex.ru/}{yandex.ru} \\
& Software Engineer \\
& \begin{itemize}[topsep = 0pt, itemsep = 0pt]
    \item[--] Поисковый антиспам
    \item[--] Разработка распределенной системы хранения данных
    \item[--] Разработка и оптимизация производительности системы обработки потока web--документов
\end{itemize}
\\

2011 --- 2013 & \textbf{``Лаборатория трехмерного зрения''} \\
& \href{http://www.3detection.com/}{www.3detection.com} \\
& Junior C / C++ Developer \\
& \begin{itemize}[topsep = 0pt, itemsep = 0pt]
    \item[--] Разработал систему автоматического управления
    3х степенного манипулятора на основе микроконтроллеров семейства STM8.
    \item[--] ПО для комплексного тестирования, настройки и диагностики
    неисправностей для узлов и систем манипулятора робота.
    \item[--] Занимался работой над системой навигации робота,
    написал алгоритм автоматической стыковки робота с док--станцией.
    \item[--] Поддержка ПО на платформу x86 написанного на C++, Lua 5.1
    для управления, отладки и контроля за микроконтроллерными системами.
    \item[--] Сделал систему технического зрения для
    распознавания интерьеров по видеопотоку на основе (OpenCV 2.3).
    \item[--] Работал над искусственным интеллектом робототехнического комплекса.
    (проект был заморожен).
\end{itemize}
\\
\end{longtable}

\section{Образование}
\begin{longtable}{p{20mm}|p{140mm}}
2008 --- 2014
& ``МГТУ им Н.Э. Баумана'' \\
& \href{http://www.bmstu.ru}{www.bmstu.ru} \\
& Факультет ``Специальное машиностроение'' \\
& Кафедра ``Специальная робототехника и мехатроника'' \\
& Полученная степень: Специалист \\
& Год окончания 2014 \\
& Диплом с отличием \\
& Тема дипломного проекта: ``\textit{Привод для двухстепенного манипулятора спутника}'' \\
& Научный руководитель: Бошляков Андрей Анатольевич \\
\end{longtable}

% \section{Научные и фундаментальные знания}
% \begin{itemize}
%     \item Системы автоматического управления
%     \item Алгоритмы и структуры данных, линейная алгебра, дискретная математика
%     \item Теоретическая механика
% \end{itemize}

\section{Иностранные языки}
\begin{itemize}
    \item[--] Русский    --- родной язык
    \item[--] Английский --- Intermediate
\end{itemize}


\end{document}

