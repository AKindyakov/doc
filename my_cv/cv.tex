\documentclass[a4paper, 11pt]{article}
%\documentclass{article}
\usepackage [utf8] {inputenc}
% включаем переносы для русского и английского языка
\usepackage[english,russian]{babel}
% Начинать первый параграф раздела следует с красной строки
\usepackage{indentfirst}
% Выбор внутренней TEX−кодировки
\usepackage [T2A]{fontenc}

\usepackage{cmap}
\usepackage{multirow}

\usepackage{xcolor}

\usepackage{hyperref}
\definecolor{LINKCOLOUR}{rgb}{0.1,0.0,0.9}
\hypersetup{colorlinks,breaklinks,urlcolor=LINKCOLOUR,linkcolor=LINKCOLOUR}

\usepackage{geometry} % Меняем поля страницы
\geometry{left=2.5cm}% левое поле
\geometry{right=2cm}% правое поле
\geometry{top=2cm}% верхнее поле
\geometry{bottom=2cm}% нижнее поле

\usepackage{titlesec} % Used to customize the \section command
\titleformat{\section}{\Large\raggedright}{\color{green}}{0em}{}[\titlerule] % Text formatting of sections
%\titleformat{\section}{\large\scshape\raggedright}{\color{green}}{0em}{}[\titlerule] % Text formatting of sections
\titlespacing{\section}{2pt}{3pt}{3pt} % Spacing around sections

\pagestyle{empty}

\begin{document}

{\LARGE\textbf{Александр Киндяков}}

\begin{flushright}
    {\itshape
        г.Москва, Россия \\
        +7-926-453-25-30 \\
        \href{mailto:akindyakov@gmail.com}{akindyakov@gmail.com} \\
        \href{https://github.com/akindyakov}{github.com/akindyakov} \\
        \href{http://ru.linkedin.com/in/alexanderkindyakov}{ru.linkedin.com/in/alexanderkindyakov} \\
    }
\end{flushright}


\section{Навыки}
\begin{itemize}
    \item Языки программирования:   C(C99), C++(ISO/IEC 2003), Lua(5.1, 5.2), Bash(UNIX Shell)
    \item Библиотеки:               STL, OpenCV(2.2, 2.3), Boost (4.8, 4.9)
    \item Прикладные программы:     CMake, GNU-Make, Vim, Doxygen, SciLab(5.3.2 - 5.4.1), Matlab
    \item Системы контроля версий:  Git, Subversion
    \item Операционные системы:     GNU/Linux(Ubintu 10.04 и выше), Windows(XP, Win7)
\end{itemize}

\section{Опыт работы }
\begin{tabular}{p{25mm}|p{110mm}}
2011 - 2013 & \textbf{``Лаборатория трехмерного зрения''} \\
& Junior C / C++ Developer \\
до настоящего момента &
\begin{itemize}
    \item Разработал систему автоматического управления
    3х степенного манипулятора на основе микроконтролеров семейства STM8.
    \item ПО для комплексного тестирования, настройки и диагностики
    неисправностей для узлов и систем манипуляторра робота.
    \item Занимался работой над системой навигации робота,
    написал алгоритм автоматической стыковки робота с докстанцией.
    \item Поддержка ПО на платформу x86 написанного на C++, Lua 5.1
    для управления, отладки и контроля за микроконтролерными системами.
    \item Сделал систему технического зрения для
    распознавания интерьеров по видеопотоку на основе (OpenCV 2.3).
    \item Учавствовал разработке систем стабилизации и
    распознования входного видеопотока.
    \item Работал над искуственным интеллектом роботехнического комплекса.
    (проект был заморожен)
\end{itemize}
\\
\end{tabular}

\section{Образование}
\begin{tabular}{p{25mm}|p{110mm}}
2008-2014             & ``МГТУ им Н.Э. Баумана'' \\
до настоящего момента & Факультет ``Специальное машиностроение'' \\
                      & Кафедра ``Специальная робототехника и мехатроника'' \\
                      & URL: \href{http://bmstu.ru}{bmstu.ru} \\
                      & Получаемая степень: Магистр \\
                      & Год окончания 2014
\end{tabular}

\section{Научные и фундаментальные знания}
\begin{itemize}
    \item Системы автоматического управления
    \item Алгоритмы и структуры данных, линейная алгебра, дискретная математика
    \item Теоретическая механика
\end{itemize}

\section{Иностранные языки}
\begin{itemize}
    \item Русский    - родной язык
    \item Английский - Pre-Intermediate
\end{itemize}


\section{Интересы}
\begin{itemize}
    \item Алгоритмы и методики в области технического зрения, экпериметировал с OpenCV.
    \item Классические алгоритмы и ихреализация, разбираю книгу P. Седжвика, ``Алгоритмы C++''
    \item Литература: читаю книги, как о разработке и проектировании програмных
продуктов так и художественные.

\href{http://www.goodreads.com/user/show/24404721-alexander-kindyakov}{goodreads.com/user/show/alexander-kindyakov}
    \item Разработка игр: в компании с моим другом разрабатываем небольшую игру, в которой
используем C++, STL, Polycode Framework и CMake в качестве системы сборки.

\href{https://github.com/thegreenbox/helium}{GitHub.com/TheGreenBox/Helium}
    \item Спорт: спортивные бальные танцы и плавание
\end{itemize}

\end{document}

